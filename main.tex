%
% LaTeX2e template for FIT2010
%

%本テンプレートは *非公式* のものです.
%ご自身の責任においてご利用下さい.

\documentclass[a4j,twocolumn]{jarticle}
\usepackage{url}
\usepackage{amsmath,amssymb,amsfonts,amsthm,enumerate}
%\usepackage[dvips]{graphicx}

\makeatletter
% \sectionと\subsection部分のフォントサイズを10.5ポイントに設定します
% 参考URL http://www.h4.dion.ne.jp/~latexcat/intro/i7-r2.html
\def\section{\@startsection{section}{1}{\z@}%
   {0.6\Cvs}%
   {0.3\Cvs}%
   {\reset@font\fontsize{10.5pt}{0pt}\bfseries}}
\def\subsection{\@startsection{subsection}{2}{\z@}%
   {\Cdp}%
   {\Cdp}%
   {\reset@font\fontsize{10.5pt}{0pt}\bfseries}}
% 章番号の後にピリオドを入れます.
% 参考URL http://oku.edu.mie-u.ac.jp/~okumura/texfaq/qa/34044.html
\def\@seccntformat#1{\csname the#1\endcsname.}
\def\thefootnote{\fnsymbol{footnote}}
\makeatother


\def\baselinestretch{0.83}

% 各種マージンの設定
% 参考URL http://www.nsknet.or.jp/~tony/TeX/faq/layout.htm
\setlength{\oddsidemargin}{20mm}
\setlength{\evensidemargin}{20mm}
\addtolength{\oddsidemargin}{-1in}% デフォルトのマージンを引きます.
\addtolength{\evensidemargin}{-1in}% デフォルトのマージンを引きます.
\setlength{\textwidth}{170mm}% 210-20-20
\setlength{\topmargin}{30mm}
\addtolength{\topmargin}{-1in}% デフォルトのマージンを引きます.
\setlength{\headheight}{0mm}
\setlength{\headsep}{0mm}
\setlength{\textheight}{242mm}% 297-30(top)-25(bottom)
\setlength{\columnsep}{7mm}

% local settings
% end of local settings

\begin{document}
\pagestyle{empty}
\thispagestyle{empty}

\twocolumn[%
\begin{center}
 {\Large 拡張可能なドキュメント検査ツール RedPen}\vspace{.5ex}

 {\Large\sffamily Extensible Document Validation Tool, RedPen}\vspace{1ex}

\large
\mbox{}
\hfil
\setcounter{footnote}{2}
{\bfseries 伊藤敬彦}${}^\thefootnote$
\hfil
\mbox{}

\mbox{}
\hfil
{\sffamily Takahiko Ito}
\hfil
\mbox{}
\hfil

\end{center}
]

\setcounter{footnote}{2}
\footnotetext{株式会社リクルートテクノロジーズ}

%これを入れると行・文字が詰まります.
%\fontsize{9pt}{0pt}\selectfont

\section{概要}

ソフトウェアエンジニアや研究者には、マニュアルや論文などの技術文書を書く機会が多く存在する.
技術文書は ``規約'' にしたがって記述するという共通の特徴を持つ.
文書の規約は文書の執筆者が従うべきルールである.

一般に規約は集団で文書を作成する際にメンバが従うべき共通のルールとして使用される.
個人で文書を記述する際にも,文書全体が一貫した記述になるために策定される.
規約には一文の長さ,利用する句読点の種類(半角全角など),文書中で利用する技術単語の選択などがある.
規約は文書を作成する組織ごとに大きく異なる.たとえば,アルゴリズムをアルファベットで記述する組織もあれば,
カタカナに変換して記述する組織も存在する.どちらを採用しても大きな問題はないが,
規約が混在してしまうと文書の可読性が低下したり,印象を損ねるおそれがある.

そのため、規約の遵守は重要な課題の一つと言える。本稿ではドキュメントが規約に従って記述されたか
自動検査するツール RedPen について解説する。

次節でドキュメントを自動で検査するツール(ドキュメント検査ツール)について紹介する。
その後 RedPen の特徴と拡張方法について解説する。

\section{背景: ドキュメント検査ツール}
これまでにドキュメント検査ツールは提案されてきた。
株式会社ジャストシステムが提供している文書校正支援ツール Just Right!~\cite{justright} は
文の誤り検査(誤字脱字,仮名遣い,慣用表現,呼応表現,ら抜き表現,同音語誤り,二重敬語),用語基準(送り仮名,漢字基準),表現
(文体の統一,重ね言葉,同一助詞の連続,二重否定)など多くの機能を提供している.
ただし Just Right! は商用製品のため無料で利用できない.

また自動で文書検査するツールに日本語表現法開発プロジェクト(PaWeL)が公開している Tomarigi~\cite{tomarigi}~\cite{tomarigi-paper}
は無料で利用できる文書の自動検査ツールや``Chantokun''~\cite{chantokun} がある.しかしこれらのツールはコマンドラインでの利用ができない.
そのため,ソフトウェアエンジニアが Git などのレポジトリ管理ツールや他のコマンドラインツール群と組み合わせて利用するのが難しい.また,
ユーザや所属組織によって異なる規約にフィットするための規約の更新方法がシステムに提供されていない. 

文法誤りの検出だけではなく訂正を行う研究に水本ら~\cite{mizumoto12english} の英文法自動誤り訂正を行ったものがあるが,手法は一般的に利用
できる形では配布されていない.

\section{RedPenの特徴}

\section{RedPenの拡張}

\section{まとめ}

\bibliographystyle{plain}
\bibliography{reference-j,reference}

\end{document}
